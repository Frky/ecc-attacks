\documentclass{beamer}
%\documentclass[handout]{beamer}

\usetheme{ENSIMAG}

\definecolor{redtrefle}{rgb}{0.5, 0, 0}
\definecolor{redmoka}{rgb}{0.25, 0.5, 0}
\definecolor{redminuit}{rgb}{0.5, 0, 0.5}
\definecolor{lightgray}{rgb}{0.6, 0.6, 0.6}
\definecolor{dkgreen}{rgb}{0,0.6,0}
\definecolor{gray}{rgb}{0.5,0.5,0.5}
\definecolor{mauve}{rgb}{0.58,0,0.82}
\definecolor{orange}{rgb}{1,0.45,0}
\definecolor{dkblue}{rgb}{0.098,0.098,0.44}

\usepackage[utf8]{inputenc}
\usepackage[french]{babel}
\usepackage{color}
\usepackage{verbatim}
\usepackage{graphicx}
\usepackage{listings}

\usepackage[absolute,showboxes,overlay]{textpos}

%\TPshowboxestrue
\TPshowboxesfalse
\textblockorigin{10mm}{10mm} % origine des positions


% command to highlight text in orange
\newcommand{\alertor}[1]{\textcolor{orange}{#1}}

\usepackage{tikz}
\tikzset{visib/.style={rectangle, color=black, fill=gray!10, draw, text=black, text opacity=1, text width=#1,align=flush center}}
\tikzset{rvisib/.style={text=black,text opacity=1, text width=#1,align=flush right}}
\tikzset{invisib/.style={rectangle,color=gray,fill=gray!10,text=black,draw,text opacity=0.4, text width=#1,align=flush center}}
\newenvironment{myfancyblock}%
{\begin{center}\begin{tikzpicture}}%
{\end{tikzpicture}\end{center}}%
\newcommand{\opaqueblock}[4]{
    \node[#2=#3] (X) {#4};
}

% Settings pour l'insertion de code %
\lstset{
  language=C,
  aboveskip=3mm,
  belowskip=3mm,
  xleftmargin=9mm,
  showstringspaces=false,
  columns=flexible,
  basicstyle={\tiny\ttfamily},
  numbers=left,
  numberstyle=\tiny\ttfamily\color{mauve},
  keywordstyle=\color{orange},
  commentstyle=\color{dkblue},
  stringstyle=\color{mauve},
  breakatwhitespace=true
  tabsize=4
}

\lstset{escapechar=\§}

\title[ECC Side channel attacks]{Side Channel attacks on ECC}

\subtitle{Review of possible attacks}

\author{Franck De Goër\\Guillaume Jeanne}

\institute{SCCI - Ensimag}

\date{January, 2014}

\AtBeginSection[]
{
    \begin{frame}<beamer>
        \frametitle{Overview}
        \tableofcontents[currentsection]
    \end{frame}
}

\begin{document}

\begin{frame}
    \titlepage
\end{frame}

\begin{frame}[t]
    \frametitle{Introduction}
    \vspace{1cm}
    \begin{description}
        \item[{\bf Subjet}] Side channel attacks over ECC algorithms.
        \item[{\bf Goal}] Perform attacks to recover a secret key from a cryptoraphic algorithm using elliptic curve. \\
	Make an overview of different methods and detail some of them that have an important impact on ECC cryptography.
	
    \end{description}
\end{frame}

\begin{frame}
    \frametitle{Overview}
    \tableofcontents
\end{frame}

\begin{frame}
    \frametitle{What is a side channel attack ?}
    \begin{itemize}
        \item \textcolor{black} {take advantage of the interaction between a device and its environment.}
        \item \textcolor{black} {apply to the embedded world: smartcards}
	\item \textcolor{black} {All environement variables: time, power consumption, electromagnetic field, acoustic...}
    \end{itemize}

\end{frame}


\section{The double-and-add algorithm}
    \subsection{Universality}
    \subsection{Importance}

\section{Non-invasive attacks}

\begin{frame}
    \frametitle{Non-invasive attacks}
    \begin{itemize}
        \item \textcolor{black} {Do not destroy or alter the functioning of the device}
        \item \textcolor{black} {Cannot be detected by the owner of the device}
    \end{itemize}

\end{frame}

    \subsection{Timing attacks}
    \subsection{Power analysis}
        \subsubsection{SPA: Simple power analysis}
	\subsubsection{SPA: Differential power analysis}

\section{Perturbation attacks}
    \subsection{Naive attack}
    \subsection{During the computation}

\begin{frame}
    \frametitle{Conclusion}

    \begin{itemize}
        \item \textcolor{black}{These attacks are efficient and not difficult to set up}
        \item \textcolor{black}{All countermeasures increase the cost of the algorithm}
	\item \textcolor{black}{Effectiveness depend on the hardware precision of the attacker}
    \end{itemize}

\end{frame}


\begin{frame}
    \frametitle{}
    \begin{center}
        {\Huge\bf\tt Thank you for your attention.}
    \end{center}
\end{frame}


%\section{Début 2}
%
%\subsection{Tout début}
%
%\begin{frame} \frametitle{Compilation de slides}
%\begin{myfancyblock}
%% First block
%\opaqueblock{1}{\textwidth}{All you have to do to initialize a GLSurfaceView is call setRenderer().
%However, if desired, you can modify the default behavior of GLSurfaceView
%by calling \alertor{one or more} of these methods before \alertor{setRenderer}:
%\begin{itemize}
%\item setDebug()
%\item setChooser()
%\item setWrapper()
%\end{itemize}
%\begin{flushright}
%(Android    2.2 API Reference)
%\end{flushright}
%}
%\invblock{2-}{\textwidth}
%
%% Second block
%\opaqueblock{2}{0.6\textwidth}{You can optionally modify the behaviour of GLSurfaceView by calling one or more debugging methods \alertor{setDebug()}, and \alertor{setWrapper()}. These methods can be called \alertor{before and or after setRender}}
%\invblock{3-}{0.6\textwidth}
%
%% Third block
%\opaqueblock{3}{0.7\textwidth}{Once the render is set, you can control whether the render draws continuously or on demand by calling \alertor{setRenderMode()}}
%
%\end{myfancyblock}
%\end{frame}

\end{document}


%%% Local Variables: 
%%% mode: latex
%%% TeX-master: t
%%% End: 
